\documentclass[]{article}
\usepackage{lmodern}
\usepackage{amssymb,amsmath}
\usepackage{ifxetex,ifluatex}
\usepackage{fixltx2e} % provides \textsubscript
\ifnum 0\ifxetex 1\fi\ifluatex 1\fi=0 % if pdftex
  \usepackage[T1]{fontenc}
  \usepackage[utf8]{inputenc}
\else % if luatex or xelatex
  \ifxetex
    \usepackage{mathspec}
  \else
    \usepackage{fontspec}
  \fi
  \defaultfontfeatures{Ligatures=TeX,Scale=MatchLowercase}
\fi
% use upquote if available, for straight quotes in verbatim environments
\IfFileExists{upquote.sty}{\usepackage{upquote}}{}
% use microtype if available
\IfFileExists{microtype.sty}{%
\usepackage{microtype}
\UseMicrotypeSet[protrusion]{basicmath} % disable protrusion for tt fonts
}{}
\usepackage{hyperref}
\hypersetup{unicode=true,
            pdfborder={0 0 0},
            breaklinks=true}
\urlstyle{same}  % don't use monospace font for urls
\usepackage{graphicx,grffile}
\makeatletter
\def\maxwidth{\ifdim\Gin@nat@width>\linewidth\linewidth\else\Gin@nat@width\fi}
\def\maxheight{\ifdim\Gin@nat@height>\textheight\textheight\else\Gin@nat@height\fi}
\makeatother
% Scale images if necessary, so that they will not overflow the page
% margins by default, and it is still possible to overwrite the defaults
% using explicit options in \includegraphics[width, height, ...]{}
\setkeys{Gin}{width=\maxwidth,height=\maxheight,keepaspectratio}
\IfFileExists{parskip.sty}{%
\usepackage{parskip}
}{% else
\setlength{\parindent}{0pt}
\setlength{\parskip}{6pt plus 2pt minus 1pt}
}
\setlength{\emergencystretch}{3em}  % prevent overfull lines
\providecommand{\tightlist}{%
  \setlength{\itemsep}{0pt}\setlength{\parskip}{0pt}}
\setcounter{secnumdepth}{0}
% Redefines (sub)paragraphs to behave more like sections
\ifx\paragraph\undefined\else
\let\oldparagraph\paragraph
\renewcommand{\paragraph}[1]{\oldparagraph{#1}\mbox{}}
\fi
\ifx\subparagraph\undefined\else
\let\oldsubparagraph\subparagraph
\renewcommand{\subparagraph}[1]{\oldsubparagraph{#1}\mbox{}}
\fi

%%% Use protect on footnotes to avoid problems with footnotes in titles
\let\rmarkdownfootnote\footnote%
\def\footnote{\protect\rmarkdownfootnote}

%%% Change title format to be more compact
\usepackage{titling}

% Create subtitle command for use in maketitle
\providecommand{\subtitle}[1]{
  \posttitle{
    \begin{center}\large#1\end{center}
    }
}

\setlength{\droptitle}{-2em}

  \title{}
    \pretitle{\vspace{\droptitle}}
  \posttitle{}
    \author{}
    \preauthor{}\postauthor{}
    \date{}
    \predate{}\postdate{}
  
\usepackage{amsthm}
\usepackage{xcolor}

\begin{document}

\theoremstyle{break}
\newtheorem{auf}{Aufgabe}

\newcommand{\R}{{\sffamily R} }

\begin{centering}
%\vspace{-2 cm}
\Huge
{\bf ?bung 2}\\
\Large
Explorative Datenanalyse und Visualisierung\\
\normalsize
Wintersemester 2019\\
S. D?hler (FBMN, h\_da)\\
\end{centering}

\hrulefill

\textbf{Name: }

\hrulefill

\setcounter{auf}{2}
\begin{auf}

Erzeugen Sie in {\sffamily R} die Folge:
{\ttfamily 1 2 3 1 2 3 1 2 3 1 2 3}
\newline
\newline
(Folgende {\sffamily R} -Befehle k?nnen hilfreich sein: {\ttfamily rep, seq})

\end{auf}

\paragraph{L?sung}

\paragraph{Anmerkungen/Korrektur}

\textcolor{gray}{\hrulefill}

\begin{auf}
Compute using {\sffamily R} the 0.99 quantile of the standard normal distribution.
\end{auf}

\paragraph{L?sung}

\paragraph{Anmerkungen/Korrektur}

\textcolor{gray}{\hrulefill}

\begin{auf}
Simulate 100 observations from $N(50,16)$. Plot the empirical distribution function for this sample and overlay the true distribution function $F$.
\newline
\newline
(Folgender {\sffamily R} -Befehle kann hilfreich sein: {\ttfamily ecdf})
\end{auf}

\paragraph{L?sung}

\paragraph{Anmerkungen/Korrektur}

\textcolor{gray}{\hrulefill}

\begin{auf}
Schreiben Sie eine {\sffamily R} -Funktion {\ttfamily diff.med}, die zu einem gegebenen Vektor $x$ die Differenz des median und des arithmetischen Mittels ausgibt.
\begin{enumerate}
    \item[(a)] Simulieren Sie 100 Beobachtungen von $N(50,16)$ und wenden Sie {\ttfamily diff.med} auf diesen Vektor an.
    \item[(b)] Simulieren Sie 100 Beobachtungen von $LN(1,2)$ (log-normal) und wenden Sie {\ttfamily diff.med} auf diesen Vektor an.
    \end{enumerate}
Vergleichen Sie die beiden Ergebnisse. Was w?rden Sie erwarten? F?hren Sie ggf. mehr Simulationen durch und untersuchen Sie die Histogramme, die zu den Daten aus (a) und (b) geh?ren.
\\
(Folgende {\sffamily R} -Befehle k?nnen hilfreich sein: {\ttfamily mean, median, hist})
\end{auf}

\paragraph{L?sung}

\paragraph{Anmerkungen/Korrektur}

\textcolor{gray}{\hrulefill}

\begin{auf}
Ein Risiko $V$ h?ngt von einem anderen Risiko $X$ folgenderma?en ab:
\[
 V=e^X+2 X^2 \qquad \text{mit} \qquad X \sim N(0,1).
 \]
Schreiben Sie eine {\sffamily R} -Funktion {\ttfamily Sim.V}, die zu einem gegebenen Niveau $\alpha \in (0,1)$ und gegebener Anzahl $N_{Sim}$ von Simulationen von $V$ das $\alpha$-Quantil der Verteilung von $V$ sch?tzt.
\newline
\newline
(Folgende {\sffamily R} -Befehle k?nnen hilfreich sein: {\ttfamily quantile})
\end{auf}

\paragraph{L?sung}

\paragraph{Anmerkungen/Korrektur}


\end{document}
